% Алгоритм возведения в целую степень x1^x2

\begin{tikzpicture}[auto]
    \node [cloud] (start) {\hbox{Начало($x_1$, $x_2$)}};

    \node [block, below of=start, minimum width=3.0cm] (init) {\hbox{$result = 1$}};
    \path [line] (start) -- (init);

    \node [decision, below of=init] (loop_condition) {$x_2=0$?};
    \path [line] (init) -- (loop_condition);

    \node [cloud, left of=loop_condition, node distance=3.5cm] (halt) {\hbox{Конец($result$)}};
    \path [line] (loop_condition) -- node[near start, above] {да} (halt);

    \node [decision, below of=loop_condition, node distance=3.0cm] (inloop_cond) {$x_2 \vdots 2$?};
    \path [line] (loop_condition) -- node[near start, above]  {нет}(inloop_cond);

    \node [block, below left of=inloop_cond, minimum width=4.2cm, node distance=3.2cm] (inloop_yes) {\hbox{$x_2 = x_2/2$} \hbox{$x_1 = x_1*x_1$}};
    \path [line] (inloop_cond) -| node[near start, above]  {да}(inloop_yes);

    \node [block, below right of=inloop_cond, minimum width=4.2cm, node distance=3.2cm] (inloop_no) {\hbox{$x_2 = x_2-1$} \hbox{$result = result \cdot x_1$}};
    \path [line] (inloop_cond) -| node[near start, above]  {нет}(inloop_no);

    % возвращаемся на условие цикла
    \path[draw] (inloop_no.south) -- ++ (0, -0.5) coordinate (N2) -| coordinate (N1) (inloop_yes.south);
    \draw ($ (N1) ! .5 ! (N2) $) -- ++ (0, -0.5) -| ($(inloop_no.east) + (0.7,0)$) [line] |- (loop_condition.east) ;

\end{tikzpicture}

\vspace{1cm}

\begin{tabular}[c]{c|c|c|c|c|c}
\textbf{№} & \textbf{действие} & $x_1$ & $x_2$ & \textbf{$result$}\\\hline
1 & Начало & 4 & 3 & & \\\hline
  &        &   &   & & \\\hline
  &        &   &   & & \\\hline
  &        &   &   & & \\\hline
  &        &   &   & & \\\hline
  &        &   &   & & \\\hline
  &        &   &   & & \\\hline
  &        &   &   & & \\\hline
  &        &   &   & & \\\hline
  &        &   &   & & \\\hline
  &        &   &   & & \\\hline
  &        &   &   & & \\\hline
  &        &   &   & & \\\hline
  &        &   &   & & \\
\end{tabular}
